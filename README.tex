% Options for packages loaded elsewhere
\PassOptionsToPackage{unicode}{hyperref}
\PassOptionsToPackage{hyphens}{url}
\PassOptionsToPackage{dvipsnames,svgnames,x11names}{xcolor}
%
\documentclass[
  11pt,
  english,
  singlespacing,
  headsepline]{MastersDoctoralThesis}

\usepackage{amsmath,amssymb}
\usepackage{iftex}
\ifPDFTeX
  \usepackage[T1]{fontenc}
  \usepackage[utf8]{inputenc}
  \usepackage{textcomp} % provide euro and other symbols
\else % if luatex or xetex
  \usepackage{unicode-math}
  \defaultfontfeatures{Scale=MatchLowercase}
  \defaultfontfeatures[\rmfamily]{Ligatures=TeX,Scale=1}
\fi
\usepackage{lmodern}
\ifPDFTeX\else  
    % xetex/luatex font selection
\fi
% Use upquote if available, for straight quotes in verbatim environments
\IfFileExists{upquote.sty}{\usepackage{upquote}}{}
\IfFileExists{microtype.sty}{% use microtype if available
  \usepackage[]{microtype}
  \UseMicrotypeSet[protrusion]{basicmath} % disable protrusion for tt fonts
}{}
\makeatletter
\@ifundefined{KOMAClassName}{% if non-KOMA class
  \IfFileExists{parskip.sty}{%
    \usepackage{parskip}
  }{% else
    \setlength{\parindent}{0pt}
    \setlength{\parskip}{6pt plus 2pt minus 1pt}}
}{% if KOMA class
  \KOMAoptions{parskip=half}}
\makeatother
\usepackage{xcolor}
\usepackage[paper=a4paper,inner=2.5cm,outer=3.8cm,bindingoffset=.5cm,top=1.5cm,bottom=1.5cm]{geometry}
\setlength{\emergencystretch}{3em} % prevent overfull lines
\setcounter{secnumdepth}{-\maxdimen} % remove section numbering
% Make \paragraph and \subparagraph free-standing
\makeatletter
\ifx\paragraph\undefined\else
  \let\oldparagraph\paragraph
  \renewcommand{\paragraph}{
    \@ifstar
      \xxxParagraphStar
      \xxxParagraphNoStar
  }
  \newcommand{\xxxParagraphStar}[1]{\oldparagraph*{#1}\mbox{}}
  \newcommand{\xxxParagraphNoStar}[1]{\oldparagraph{#1}\mbox{}}
\fi
\ifx\subparagraph\undefined\else
  \let\oldsubparagraph\subparagraph
  \renewcommand{\subparagraph}{
    \@ifstar
      \xxxSubParagraphStar
      \xxxSubParagraphNoStar
  }
  \newcommand{\xxxSubParagraphStar}[1]{\oldsubparagraph*{#1}\mbox{}}
  \newcommand{\xxxSubParagraphNoStar}[1]{\oldsubparagraph{#1}\mbox{}}
\fi
\makeatother

\usepackage{color}
\usepackage{fancyvrb}
\newcommand{\VerbBar}{|}
\newcommand{\VERB}{\Verb[commandchars=\\\{\}]}
\DefineVerbatimEnvironment{Highlighting}{Verbatim}{commandchars=\\\{\}}
% Add ',fontsize=\small' for more characters per line
\usepackage{framed}
\definecolor{shadecolor}{RGB}{241,243,245}
\newenvironment{Shaded}{\begin{snugshade}}{\end{snugshade}}
\newcommand{\AlertTok}[1]{\textcolor[rgb]{0.68,0.00,0.00}{#1}}
\newcommand{\AnnotationTok}[1]{\textcolor[rgb]{0.37,0.37,0.37}{#1}}
\newcommand{\AttributeTok}[1]{\textcolor[rgb]{0.40,0.45,0.13}{#1}}
\newcommand{\BaseNTok}[1]{\textcolor[rgb]{0.68,0.00,0.00}{#1}}
\newcommand{\BuiltInTok}[1]{\textcolor[rgb]{0.00,0.23,0.31}{#1}}
\newcommand{\CharTok}[1]{\textcolor[rgb]{0.13,0.47,0.30}{#1}}
\newcommand{\CommentTok}[1]{\textcolor[rgb]{0.37,0.37,0.37}{#1}}
\newcommand{\CommentVarTok}[1]{\textcolor[rgb]{0.37,0.37,0.37}{\textit{#1}}}
\newcommand{\ConstantTok}[1]{\textcolor[rgb]{0.56,0.35,0.01}{#1}}
\newcommand{\ControlFlowTok}[1]{\textcolor[rgb]{0.00,0.23,0.31}{\textbf{#1}}}
\newcommand{\DataTypeTok}[1]{\textcolor[rgb]{0.68,0.00,0.00}{#1}}
\newcommand{\DecValTok}[1]{\textcolor[rgb]{0.68,0.00,0.00}{#1}}
\newcommand{\DocumentationTok}[1]{\textcolor[rgb]{0.37,0.37,0.37}{\textit{#1}}}
\newcommand{\ErrorTok}[1]{\textcolor[rgb]{0.68,0.00,0.00}{#1}}
\newcommand{\ExtensionTok}[1]{\textcolor[rgb]{0.00,0.23,0.31}{#1}}
\newcommand{\FloatTok}[1]{\textcolor[rgb]{0.68,0.00,0.00}{#1}}
\newcommand{\FunctionTok}[1]{\textcolor[rgb]{0.28,0.35,0.67}{#1}}
\newcommand{\ImportTok}[1]{\textcolor[rgb]{0.00,0.46,0.62}{#1}}
\newcommand{\InformationTok}[1]{\textcolor[rgb]{0.37,0.37,0.37}{#1}}
\newcommand{\KeywordTok}[1]{\textcolor[rgb]{0.00,0.23,0.31}{\textbf{#1}}}
\newcommand{\NormalTok}[1]{\textcolor[rgb]{0.00,0.23,0.31}{#1}}
\newcommand{\OperatorTok}[1]{\textcolor[rgb]{0.37,0.37,0.37}{#1}}
\newcommand{\OtherTok}[1]{\textcolor[rgb]{0.00,0.23,0.31}{#1}}
\newcommand{\PreprocessorTok}[1]{\textcolor[rgb]{0.68,0.00,0.00}{#1}}
\newcommand{\RegionMarkerTok}[1]{\textcolor[rgb]{0.00,0.23,0.31}{#1}}
\newcommand{\SpecialCharTok}[1]{\textcolor[rgb]{0.37,0.37,0.37}{#1}}
\newcommand{\SpecialStringTok}[1]{\textcolor[rgb]{0.13,0.47,0.30}{#1}}
\newcommand{\StringTok}[1]{\textcolor[rgb]{0.13,0.47,0.30}{#1}}
\newcommand{\VariableTok}[1]{\textcolor[rgb]{0.07,0.07,0.07}{#1}}
\newcommand{\VerbatimStringTok}[1]{\textcolor[rgb]{0.13,0.47,0.30}{#1}}
\newcommand{\WarningTok}[1]{\textcolor[rgb]{0.37,0.37,0.37}{\textit{#1}}}

\providecommand{\tightlist}{%
  \setlength{\itemsep}{0pt}\setlength{\parskip}{0pt}}\usepackage{longtable,booktabs,array}
\usepackage{calc} % for calculating minipage widths
% Correct order of tables after \paragraph or \subparagraph
\usepackage{etoolbox}
\makeatletter
\patchcmd\longtable{\par}{\if@noskipsec\mbox{}\fi\par}{}{}
\makeatother
% Allow footnotes in longtable head/foot
\IfFileExists{footnotehyper.sty}{\usepackage{footnotehyper}}{\usepackage{footnote}}
\makesavenoteenv{longtable}
\usepackage{graphicx}
\makeatletter
\def\maxwidth{\ifdim\Gin@nat@width>\linewidth\linewidth\else\Gin@nat@width\fi}
\def\maxheight{\ifdim\Gin@nat@height>\textheight\textheight\else\Gin@nat@height\fi}
\makeatother
% Scale images if necessary, so that they will not overflow the page
% margins by default, and it is still possible to overwrite the defaults
% using explicit options in \includegraphics[width, height, ...]{}
\setkeys{Gin}{width=\maxwidth,height=\maxheight,keepaspectratio}
% Set default figure placement to htbp
\makeatletter
\def\fps@figure{htbp}
\makeatother

\usepackage[utf8]{inputenc} % Required for inputting international characters
%\usepackage[T1]{fontenc} % Output font encoding for international characters; causes problems for xelatex

%\usepackage{mathpazo} % Use the Palatino font by default

\usepackage[backend=bibtex, style=authoryear, natbib=true]{biblatex} % Use the bibtex backend with the authoryear citation style (which resembles APA)

\usepackage[autostyle=true]{csquotes} % Required to generate language-dependent quotes in the bibliography



%----------------------------------------------------------------------------------------
%	MARGINS
%----------------------------------------------------------------------------------------

\geometry{
	headheight=4ex,
	includehead,
	includefoot
}

\raggedbottom

\AtBeginDocument{
\hypersetup{pdftitle=\ttitle} % Set the PDF's title to your title
\hypersetup{pdfauthor=\authorname} % Set the PDF's author to your name
\hypersetup{pdfkeywords=\keywordnames} % Set the PDF's keywords to your keywords
}
\makeatletter
\@ifpackageloaded{caption}{}{\usepackage{caption}}
\AtBeginDocument{%
\ifdefined\contentsname
  \renewcommand*\contentsname{Table of contents}
\else
  \newcommand\contentsname{Table of contents}
\fi
\ifdefined\listfigurename
  \renewcommand*\listfigurename{List of Figures}
\else
  \newcommand\listfigurename{List of Figures}
\fi
\ifdefined\listtablename
  \renewcommand*\listtablename{List of Tables}
\else
  \newcommand\listtablename{List of Tables}
\fi
\ifdefined\figurename
  \renewcommand*\figurename{Figure}
\else
  \newcommand\figurename{Figure}
\fi
\ifdefined\tablename
  \renewcommand*\tablename{Table}
\else
  \newcommand\tablename{Table}
\fi
}
\@ifpackageloaded{float}{}{\usepackage{float}}
\floatstyle{ruled}
\@ifundefined{c@chapter}{\newfloat{codelisting}{h}{lop}}{\newfloat{codelisting}{h}{lop}[chapter]}
\floatname{codelisting}{Listing}
\newcommand*\listoflistings{\listof{codelisting}{List of Listings}}
\makeatother
\makeatletter
\makeatother
\makeatletter
\@ifpackageloaded{caption}{}{\usepackage{caption}}
\@ifpackageloaded{subcaption}{}{\usepackage{subcaption}}
\makeatother

\ifLuaTeX
  \usepackage{selnolig}  % disable illegal ligatures
\fi
\usepackage{bookmark}

\IfFileExists{xurl.sty}{\usepackage{xurl}}{} % add URL line breaks if available
\urlstyle{same} % disable monospaced font for URLs
\hypersetup{
  colorlinks=true,
  linkcolor={blue},
  filecolor={Maroon},
  citecolor={magenta},
  urlcolor={red},
  pdfcreator={LaTeX via pandoc}}


\thesistitle{} % Your thesis title, this is used in the title and abstract, print it elsewhere with \ttitle
\supervisor{} % Your supervisor's name, this is used in the title page, print it elsewhere with \supname
\examiner{} % Your examiner's name, this is not currently used anywhere in the template, print it elsewhere with \examname
\degree{Bachelor of
Science} % Your degree name, this is used in the title page and abstract, print it elsewhere with \degreename
\author{} % Your name, this is used in the title page and abstract, print it elsewhere with \authorname
\addresses{} % Your address, this is not currently used anywhere in the template, print it elsewhere with \addressname

\subject{} % Your subject area, this is not currently used anywhere in the template, print it elsewhere with \subjectname
\keywords{} % Keywords for your thesis, this is not currently used anywhere in the template, print it elsewhere with \keywordnames
\university{Johannes Kepler Universität
Linz} % Your university's name, this is used in the title page and abstract, print it elsewhere with \univname
\department{Institute of Polymer
Chemistry} % Your department's name, this is used in the title page and abstract, print it elsewhere with \deptname
\group{Biological
Chemistry} % Your research group's name and URL, this is used in the title page, print it elsewhere with \groupname
\faculty{Faculty of Engineering and Natural
Sciences} % Your faculty's name and URL, this is used in the title page and abstract, print it elsewhere with \facname

\setcounter{tocdepth}{3} % The depth to which the document sections are printed to the table of contents
\begin{document}
\frontmatter % Use roman page numbering style (i, ii, iii, iv...) for the pre-content pages

\pagestyle{plain} % Default to the plain heading style until the thesis style is called for the body content

%----------------------------------------------------------------------------------------
%	TITLE PAGE
%----------------------------------------------------------------------------------------

\begin{titlepage}
\begin{center}

% Add the logo at the top center
\includegraphics[height=3cm]{images/logo.png} % University/department logo

\vspace*{.06\textheight}
{\scshape\LARGE \univname\par}\vspace{1.5cm} % University name
\textsc{\Large BACHELOR THESIS}\\[0.5cm] % Thesis type

\HRule \\[0.4cm] % Horizontal line
{\LARGE \bfseries \ttitle\par}\vspace{0.4cm} % Thesis title, changed from \huge to \LARGE
\HRule \\[1.5cm] % Horizontal line

\begin{tabular}{@{}p{0.4\textwidth}p{0.55\textwidth}@{}}
\begin{flushleft} \large
\emph{Author:}\\
\textbf{\authorname}
\end{flushleft}
&
\begin{flushright} \large
\begin{tabular}{@{}ll@{}}
\emph{Supervisor:} & \textbf{Assoz. Univ.-Prof.~Dr.~Ian Teasdale} \\
\emph{Co-Supervisor:} & \textbf{Dipl.-Ing. Michael Kneidinger} \\
\end{tabular}
\end{flushright}
\end{tabular}\\[3cm]  

\vfill

\large \textit{Bachelor's Thesis to confer the academic degree of\\ \degreename}\\[0.3cm] % University requirement text
\textit{in}\\[0.4cm]
\textbf{\groupname}\\
\href{}{\deptname}\\[2cm] % Research group name and department name with link

\vfill


\vfill
\end{center}
\end{titlepage}

%----------------------------------------------------------------------------------------
%	DECLARATION PAGE
%----------------------------------------------------------------------------------------
\begin{declaration}
\addchaptertocentry{\authorshipname} % Add the declaration to the table of contents
\noindent I hereby declare that the thesis submitted is my own unaided work, that I have not used other than the sources indicated, and that all direct and indirect sources are acknowledged as references. This printed thesis is identical with the electronic version submitted.

 
\noindent Signed:\\
\rule[0.5em]{25em}{0.5pt} % This prints a line for the signature
 
\noindent Date:\\
\rule[0.5em]{25em}{0.5pt} % This prints a line to write the date


\end{declaration}

\cleardoublepage

%----------------------------------------------------------------------------------------
%	QUOTATION PAGE
%----------------------------------------------------------------------------------------

\vspace*{0.2\textheight}

\noindent``{\itshape We make our progress through explanatory
conjectures and criticism. And, as Popper says, by letting our ideas
`die in our place'.}''\bigbreak

\hfill David Deutsch



%----------------------------------------------------------------------------------------
%	ACKNOWLEDGEMENTS
%----------------------------------------------------------------------------------------

\begin{acknowledgements}
\addchaptertocentry{\acknowledgementname} % Add the acknowledgements to the table of contents
\input{"Frontmatter/acknowledgements.tex"}
\end{acknowledgements}


% 

\begingroup
\hypersetup{linkcolor=black}

\tableofcontents % Prints the main table of contents

\listoffigures % Prints the list of figures

\listoftables % Prints the list of tables

\endgroup


%----------------------------------------------------------------------------------------
%	ABBREVIATIONS
%----------------------------------------------------------------------------------------

\input{"Frontmatter/abbreviations.tex"}



%----------------------------------------------------------------------------------------
%	SYMBOLS
%----------------------------------------------------------------------------------------

\input{"Frontmatter/symbols.tex"}



%----------------------------------------------------------------------------------------
%	THESIS CONTENT - CHAPTERS
%----------------------------------------------------------------------------------------

\mainmatter % Begin numeric (1,2,3...) page numbering

\pagestyle{thesis} % Return the page headers back to the "thesis" style
% Define some commands to keep the formatting separated from the content 
\newcommand{\keyword}[1]{\textbf{#1}}
\newcommand{\tabhead}[1]{\textbf{#1}}
\newcommand{\code}[1]{\texttt{#1}}
\newcommand{\file}[1]{\texttt{\bfseries#1}}
\newcommand{\option}[1]{\texttt{\itshape#1}}


\section{Background}\label{background}

This git repo provides a template for writing bachelor or diploma thesis
using Quarto. Originally adapted from a LaTeX template by
\href{https://github.com/nmfs-opensci/quarto-thesis}{Eli Holmes}, I
tailored it to meet JKU's requirements. The template can also be use for
phd desertation (orginal design).

\subsection{Installing the extension}\label{installing-the-extension}

You will need to do this to get all the folders with tex files. Start in
the directory where you will create the directory that will contain your
thesis files. Run this from a terminal in that directory.

\begin{Shaded}
\begin{Highlighting}[]
\ExtensionTok{quarto}\NormalTok{ use template jallow{-}code/jku{-}quarto{-}thesis}
\end{Highlighting}
\end{Shaded}

It will ask for an empty directory name where to put the files, give it
a new directory name.

Once you do that you can cd to the new directory and render from within
the directory.

\begin{Shaded}
\begin{Highlighting}[]
\ExtensionTok{quarto}\NormalTok{ render}
\end{Highlighting}
\end{Shaded}

You may encounter the following error message when you first render the
document:

\texttt{ERROR:\ \ compilation\ failed-\ missing\ packages\ (automatic\ installed\ disabled)\ \ \ \ \ \ \ LaTeX\ Error:\ Something\textquotesingle{}s\ wrong-\/-perhaps\ a\ missing\ \textbackslash{}item.\ \ See\ the\ LaTeX\ manual\ or\ LaTeX\ Companion\ for\ explanation.\ Type\ \ H\ \textless{}return\textgreater{}\ \ for\ immediate\ help.\ \ ...\ \ l.872\ \textbackslash{}end\{CSLReferences\}}\{=\}

This have to do with your version of \texttt{Quarto} and
\texttt{Tinytex} . To fix this issue you will have to update your
tinytex package or the Quarto version you are using. See
\href{https://yihui.org/tinytex/}{here} to learn how to install or
maintain \texttt{TinyTex}.

\subsection{Installation or updating for an existing
document}\label{installation-or-updating-for-an-existing-document}

You may also use this format with an existing Quarto project or
document. But you will need to have all the tex folders already (see
above).

\begin{Shaded}
\begin{Highlighting}[]
\ExtensionTok{quarto}\NormalTok{ install extension jallow{-}code/jku{-}quarto{-}thesis}
\end{Highlighting}
\end{Shaded}

\subsection{Basic Usage and
Customization}\label{basic-usage-and-customization}

The template have bachelor thesis as a default. Therefore, to change it
to a master or Phd dissertation you will the have to edit the
\texttt{\_extensions/quarto-thesis/partials/before-body.tex} file.
Scroll down to the title page of the file and edit the line that contain
thesis type.

\%----------------------------------------------------------------------------------
\% TITLE PAGE
\%----------------------------------------------------------------------------------

\textbackslash begin\{titlepage\} \textbackslash begin\{center\}

\% Add the logo at the top center \(if(thesis.logo)\)
\(if(thesis.logo-height)\)
\includegraphics[height=$thesis.logo-height$]{$thesis.logo$} \%
University/department logo \(else\)
\includegraphics[width=0.1\textwidth]{$thesis.logo$} \% Adjust the width
to make the logo smaller \(endif\) \(endif\)

\vspace*{.06\textheight}

\{\scshape\LARGE \univname

\par

\}\vspace{1.5cm} \% University name
\textsc{\Large BACHELOR THESIS}\textbackslash{[}0.5cm{]} \% Thesis type

\HRule \textbackslash{[}0.4cm{]} \% Horizontal line
\{\LARGE \bfseries \ttitle

\par

\}\vspace{0.4cm} \% Thesis title, changed from \huge to \LARGE
\HRule \textbackslash{[}1.5cm{]} \% Horizontal line

Additionally, you may want to exclude some sections of the frontmatter
(e.g dedication, constants etc..). This can be achieved simply by
commenting out that section from the \texttt{\_quarto.yml}:

This will leave out the dedication page from the rendered documents.
Checkout the pdf file in the repo to learn more about the usage of the
template. You may have to write some \texttt{Latex} code to further
customize this template if you have other specific requirements.
\href{https://github.com/nmfs-opensci/quarto-thesis}{Eli Holmes} also
excellent video tutorials you may find useful.

\subsection{Example}\label{example}

\includegraphics{Synthesis-of-Hydrogels-by-Inclusion-Complexation-between-Poly-organo-phosphazenes-and-$α$-cyclodextrin.pdf}




\end{document}
